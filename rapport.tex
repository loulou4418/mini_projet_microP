\documentclass[a4paper]{article}
\usepackage{geometry}
\geometry{a4paper}
\usepackage[utf8]{inputenc}
\usepackage{graphicx}
\graphicspath{ {./image/} }
\usepackage{amsmath}
\usepackage{amssymb}
\usepackage{listings}
\usepackage{xcolor}
\usepackage{cite}

\definecolor{mGreen}{rgb}{0,0.6,0}
\definecolor{mGray}{rgb}{0.5,0.5,0.5}
\definecolor{mPurple}{rgb}{0.58,0,0.82}
\definecolor{backgroundColour}{rgb}{0.95,0.95,0.92}

\lstdefinestyle{CStyle}{
    backgroundcolor=\color{backgroundColour},   
    commentstyle=\color{mGreen},
    keywordstyle=\color{magenta},
    numberstyle=\tiny\color{mGray},
    stringstyle=\color{mPurple},
    basicstyle=\footnotesize,
    breakatwhitespace=false,         
    breaklines=true,                 
    captionpos=b,                    
    keepspaces=true,                 
    numbers=left,                    
    numbersep=5pt,                  
    showspaces=false,                
    showstringspaces=false,
    showtabs=false,                  
    tabsize=2,
    language=C
}          

\title{\textbf{projet} \\ document}

\author{Louison GOUY}

\begin{document}
\maketitle

\vspace{2cm}

\begin{center}
   % \includegraphics[width=12cm]{logo_tp.jpg}
\end{center}

\vspace*{1cm}

\begin{center}
    name
\end{center}

\vspace*{7cm}

\begin{center}

abstract

\end{center}
\newpage
\renewcommand{\contentsname}{Table des Matières}
\tableofcontents
\newpage
\renewcommand\listfigurename{Liste des figures}
\listoffigures
\newpage

\section{section1}

\paragraph{Affinage de system\_init}
~~\\

\noindent Au début de la fonction \texttt{main} la fonction \texttt{system\_init()} est appelée. Une telle fonction a pour but d'initialiser le système, celle-ci est décrite dans le fichier \texttt{system.c}



\begin{lstlisting}[style=CStyle]
//code here
\end{lstlisting}

\bibliography{biblio}{}
\bibliographystyle{plain}

\end{document}